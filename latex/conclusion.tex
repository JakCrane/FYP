% This report set out to investigate the behaviour of fluidised powder feed systems for in-space additive manufacturing. Building on previous work analysing cold spray additive manufacturing, a subsystem of the cold spray process was disected and areas of improvement were analysed.

% Experimental data found mass flow rates of up to 60 g/s can be found with


% This report has demonstrated that pneumatic powder feed systems are a promising option for CSAM in space. Inadvertently, the scalability of this design paradigm has been well demonstrated allowing for CSAM to be a strong candidate method to build large structures like habitats or solar arrays in space.



\section{Research Objectives and Key Findings}
This report set out to investigate and improve the powder feed mechanism for in-space cold spray additive manufacturing, addressing several specific objectives. Additive manufacturing in space has been demonstrated, but only with wire-fed systems, limiting the additive manufacturing techniques available for manufacturing in space.

First, the hypothesised shortcomings of the previous feed system design were analysed and supported by numerical simulations, justifying the need for a subsystem redesigned. 

Accordingly, a fluidized powder-bed approach with a pneumatically-driven, gas-permeable piston was chosen as the most promising architecture for feeding powder into a CSAM system. To do this, significant parallels were drawn from previous work on powder feed systems used for metallic powder fed engines and the differences between the two applications were investigated.

From this, a prototype feed-system was designed, manufactured and experimentally evaluated under terrestrial conditions to investigate controllability of mass flow, an important parameter in the cold spray process. The system demonstrated somewhat reliable powder delivery, with mass flow rate responding predictably to changes in input gas pressure. The system achieved powder feed rates on the order of 20 to 60 grams per second, substantially exceeding the throughput of prior comparable systems, which operated only at a few grams per minute. While this may limit comparisons to other work, it expands on the literature available for powder-feed systems designed for additive manufacturing, a current gap in research. The low-pressure, high mass flow rate of this system is rare for powder-feed systems and highlights how much potential the architecture still has to be realised. The results from testing provide strong evidence that the required inlet pressure for effective powder feeding can be significantly reduced, an important finding that could minimises gas consumption and hardware stress in operation. 

Another key objective was to optimise the piston geometry and understand its impact on system performance, done through an iterative design process. Early piston prototypes highlighted challenges such as an insufficient pressure force on the piston and jamming due to powder. The insights from these failures informed the final piston design: a lightweight dual-plate piston made of flexible TPU cones connected by a central rod. This compliant geometry provided the necessary resilience against jamming, and the flexibility of the TPU cones, combined with a longer piston span, also helped compensate for manufacturing imperfections allowing the piston to take up the full cross-section of the tank, increasing pressure force on the piston. 

Finally, an attempt to analyse performance of the design was conducted through simulations in both microgravity and terrestrial gravity to investigate extrapolation of Earth-based results to space. While time constraints and software limitations hampered the success of this analysis, interesting results for fluidisation in microgravity were still uncovered. A direct comparison between the two operating conditions was never made.

In conclusion, this report finds no fatal flaws in the use of fluidised powder feed systems for in-space applications but could not quite provide undeniable evidence the system would work as originally desired.

\section{Suggested Next Steps}
Owing to the broad nature of the project, many research aspects were touched upon, but not in great detail. Therefore, future work should begin by refining the fundamental experimental parameters used in system testing to enable a more precise and representative characterisation of the powder and fluidisation dynamics under microgravity conditions. 

One immediate improvement involves reducing the size of the powder particles used in the experiments. The current setup, while effective for demonstrating general system functionality, employs relatively large particle sizes that do not fully capture the nuances of granular flow in realistic applications. Finer powders are more representative of those used in additive manufacturing processes and are also more sensitive to gas drag forces, which makes their behaviour under fluidisation and compaction more dependent on subtle variations in system design. 

In addition to reducing particle size, future investigations should aim to design a system with a significantly lower mass flow rate. While the high flow rates achieved in the current system demonstrate the scalability of cold spray as a manufacturing process, they also mask smaller-scale fluctuations that could impact part quality or consistency in more delicate applications. Reducing the flow rate would enable a more detailed analysis of the feeder's repeatability and controllability, particularly in the context of long-duration additive manufacturing tasks where material must be dispensed in highly precise, uniform layers. 

Alongside these experimental refinements, further computational work is needed to optimise the geometry of the piston itself. While the current design has proven functional, it was developed largely through empirical iteration. Given that the piston strongly influences both fluidisation, by affecting the mass flow rate of gas, and compaction, by affecting the pressure differential through the system. A more rigorous modelling approach—incorporating both fluid-structure interaction and granular material behaviour, could yield significant gains in performance. 

Building on this, an important next step is the extension of numerical modelling efforts to evaluate the system's direct applicability to microgravity environments. While current simulations have provided some insight into general flow dynamics, they have not yet fully explored how the absence of gravitational forces affects powder behaviour. 

Ultimately, once the system has been refined through these experimental and numerical efforts, full validation under microgravity conditions will be essential. Testing the feeder in drop towers, parabolic flights, or suborbital vehicles would provide invaluable data on its real-world performance and confirm its readiness for integration into in-space manufacturing platforms. 


