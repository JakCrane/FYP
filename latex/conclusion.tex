% This report set out to investigate the behaviour of fluidised powder feed systems for in-space additive manufacturing. Building on previous work analysing cold spray additive manufacturing, a subsystem of the cold spray process was disected and areas of improvement were analysed.

% Experimental data found mass flow rates of up to 60 g/s can be found with


% This report has demonstrated that pneumatic powder feed systems are a promising option for CSAM in space. Inadvertently, the scalability of this design paradigm has been well demonstrated allowing for CSAM to be a strong candidate method to build large structures like habitats or solar arrays in space.



\section{Research Objectives and Key Findings}
This report set out to investigate and improve the powder feed mechanism for in-space additive manufacturing, addressing several specific objectives. 

First, the hypothesized shortcomings of the previous feed system design were supported through numerical simulations and justified the need for a subsystem redesigned. 

Accordingly, a fluidized powder-bed approach with a pneumatically-driven, permeable piston was chosen as the most promising architecture for feeding powder into a CSAM system. 

A prototype feed-system was then designed, manufactured and experimentally evaluated under terrestrial conditions to investigate controllability of mass flow. The system demonstrated somewhat reliable powder delivery, with mass flow rate responding predictably to changes in input gas pressure. The system achieved powder feed rates on the order of 20 to 60 grams per second, substantially exceeding the throughput of prior comparable systems, which operated only at a few grams per minute. While this may limit comparisons to future systems, it expands on the literature available for powder-feed systems designed for additive manufacturing, a current gap in research. Most feed systems are designed for metallic powder-fed engines and are therefore operating under slightly different considerations. Along side this, the result is strong evidence that the required inlet pressure for effective powder feeding can be significantly reduced relative to designs used in metallic powder-fed engines, an important finding that could minimizes gas consumption and hardware stress in operation. 

Another key objective was to optimize the piston geometry and understand its impact on system performance, done through an iterative design process. Early piston prototypes highlighted challenges like an insufficient pressure force on the piston and jamming due to powder. The insights from these failures informed the final piston design: a lightweight dual-plate piston made of flexible TPU cones connected by a central rod. This compliant geometry provided the necessary resilience against jamming, and the flexibility of the TPU cones, combined with a longer piston span, also helped compensate for manufacturing imperfections allowing the piston to take up the full cross-section of the tank, increasing pressure force on the piston. 

Finally, an attempt to analyse performance of the design was conducted through simulations in both microgravity and terrestrial gravity to investigate extrapolation of Earth-based results to space. While time constraints and software limitations hampered the success of this analysis, interesting results were found none the less. 

\section{Future Work}
This report 
% The two-phase flow computational model (using an Eulerian-Eulerian approach with a moving piston) provided qualitative and semi-quantitative insights. Under microgravity simulation, the system behaved as expected: a clear positive correlation emerged between piston speed and powder mass flow rate. In other words, faster piston actuation led to higher powder throughput, mirroring the trend observed in physical experiments and reported in literature. This agreement is an important validation of the simulation and suggests that the design's operating principles hold in a weightless environment. The microgravity simulations also indicated that the powder distribution within the tank remained conducive to feeding - essentially, fluidization was maintained by the gas flow even without gravity, as the design intended. To further build confidence, known behaviors from prior studies (such as the linear relationship between piston force/velocity and flow rate) were used as benchmarks for the simulation's accuracy
% file-tandjdc7fd8ydjiown6r2a
% file-tandjdc7fd8ydjiown6r2a
% . By reproducing these known phenomena in the model, we established a degree of trust in the predicted microgravity performance despite the lack of direct microgravity test data. Comparisons between the Earth-gravity and zero-gravity cases in simulation supported the expectation that the new feed system will function effectively in space, with no fundamental impediments arising in the absence of gravity. (Notably, simulations with Earth's gravity turned on were more difficult and somewhat less representative of experimental results, likely due to numerical challenges
% file-tandjdc7fd8ydjiown6r2a
% file-tandjdc7fd8ydjiown6r2a
% . However, these discrepancies were traced to modeling limitations rather than any design flaw, and they do not contradict the overall conclusion that the terrestrial results can be extrapolated to microgravity with appropriate care.) In summary, all the research objectives were met: the problems of the original design were identified and overcome through a new architecture; the new system's controllability and performance were validated in the lab; and analysis indicates that its performance can be reliably translated to a microgravity scenario.
% \section{Lessons Learned and Broader Implications}
% Beyond the quantitative results, this project provided valuable lessons and insights into the design of in-space manufacturing hardware. A foremost lesson is the complexity of granular flow control in varying gravity conditions. Initial assumptions had to be revisited in light of actual observations. For instance, it was anticipated that powder fluidization would primarily occur near the outlet cone (where the tank narrows), governed by the piston-induced pressure and gas flow. Instead, during experiments an unexpected fluidization zone was observed forming near the piston head itself
% file-tandjdc7fd8ydjiown6r2a
% . This unplanned behavior - likely caused by the specific particle size and density used - highlighted how subtle shifts in system dynamics (e.g. gas flow distribution and granular cohesion) can create alternate flow regimes. The influence of gravity was also more nuanced than expected: while the design aimed to be gravity-independent, the tests revealed that even under Earth conditions the larger powder particles experienced some gravitational settling that affected where fluidization occurred
% file-tandjdc7fd8ydjiown6r2a
% . Another observed challenge was the piston's motion. Instead of a smooth continuous descent, the piston advanced in an intermittent “stop-start” manner (discrete jumps) during some tests
% file-tandjdc7fd8ydjiown6r2a
% . This behavior was not directly apparent in the mass flow data (since powder still flowed out steadily on average), but it suggested internal stick-slip dynamics or sequential arching and collapsing of the powder inside the tank. The lesson here is that system behavior can depart from idealized models; real granular systems may exhibit stick-slip motion or multiple fluidization regions, which engineers must recognize and accommodate. By iteratively refining the design - for example, introducing flexible elements to the piston - we learned that built-in compliance and adaptability are crucial for managing the unpredictable nature of particle mechanics. In short, the development process underscored the importance of designing for robust performance under uncertain conditions, and being prepared to troubleshoot phenomena (like clogging or oscillatory motion) that only manifest in integrated testing. The project also faced and overcame several practical challenges, yielding insights for future efforts. Experimental testing in a lab environment required careful attention to measurement accuracy and repeatability. Because the available pressure sensors and load cells were not high-precision aerospace-grade devices, we adopted strategies to ensure confidence in the data. For instance, multiple load cells were deployed at different points (at the tank inlet, outlet, and under the powder collector) to provide redundant measurements of powder throughput
% file-tandjdc7fd8ydjiown6r2a
% file-tandjdc7fd8ydjiown6r2a
% . This redundancy allowed cross-validation of mass flow calculations and helped identify any aberrant readings. We also discovered the importance of consistent procedures and baseline checks (such as frequent tare measurements and monitoring of sensor drift) when using lower-cost instrumentation. Some planned diagnostics proved too complex to realize within the project scope - for example, a detailed analysis of internal pressure dynamics via transducers was intended but ultimately not completed due to time constraints and modeling difficulty
% file-tandjdc7fd8ydjiown6r2a
% . This taught a practical lesson in scoping: ambitious measurement and analysis techniques must be balanced against available time and resources, and sometimes a simpler empirical approach (or leaving an aspect for future work) is prudent. On the simulation side, the challenges were equally instructive. The advanced multi-phase, moving-mesh simulations necessary to capture the system's behavior stretched the limits of the student software license and available computational knowledge
% file-tandjdc7fd8ydjiown6r2a
% . Frequent instability and crashes forced a step-by-step simplification of the numerical model. Through this experience, we learned the value of verifying each aspect of a complex simulation (e.g., first modeling a simplified 2D case, or validating sub-models against known physical laws) before building up complexity. We also confirmed that simulation is most useful as a comparative tool rather than an absolute predictor in such novel scenarios. By comparing simulation outcomes between Earth gravity and microgravity cases (and against trends from literature), we could glean the essential effects of gravity, even if the absolute accuracy of each simulation was limited. This approach - focusing on qualitative and relative results to draw conclusions - was a necessary adaptation given the computational challenges, and it proved effective for guiding design decisions. Overall, the team's ability to navigate experimental and computational hurdles reinforced the importance of adaptability, thorough validation, and iterative problem-solving in engineering research. In terms of broader impact, the outcomes of this project bode well for the feasibility of in-space Cold Spray Additive Manufacturing (CSAM). The optimized powder feed system not only solved the specific issues of the prior design but also demonstrated performance far beyond the initial expectations. Achieving stable mass flow rates on the order of tens of grams per second is a remarkable improvement over previous systems (which managed on the order of only 3-6 grams per minute)
% file-tandjdc7fd8ydjiown6r2a
% . Such a leap in throughput could dramatically increase the deposition rates of cold spray printing in space, enabling larger or more complex structures to be built in a reasonable time frame. Importantly, this was accomplished while lowering the operational pressure requirements of the system. Our results indicate that effective feeding is possible with pressure differentials on the order of a few bar (around 3-6 bar tested) instead of ~10+ bar in earlier designs
% file-tandjdc7fd8ydjiown6r2a
% . This reduction has significant implications: a lower-pressure system can be made lighter and safer, and it consumes less propellant gas - a critical resource in space - for the same output
% file-tandjdc7fd8ydjiown6r2a
% . In short, the design improvements translate directly into mass, volume, and resource savings for a space-deployable manufacturing unit. These advantages improve the overall viability of CSAM for long-duration missions or on-orbit construction, where efficiency is paramount. Finally, the successful demonstration of a gravity-independent powder feed mechanism contributes to the general credibility of in-space manufacturing efforts. It shows that a traditionally gravity-reliant process (powder feeding) can be re-engineered for microgravity through smart use of fluidization and pressure-driven actuation. The research thus supports the notion that CSAM is a promising technology for building and repairing infrastructure in space, and it scales favorably. Indeed, the scalability of the pneumatic feeding concept - evidenced by its ability to handle high flow rates and larger powder quantities - suggests that it could be used to fabricate not just small components but even large structures (e.g. spacecraft trusses, habitat components, or solar array panels) in orbit
% file-tandjdc7fd8ydjiown6r2a
% file-tandjdc7fd8ydjiown6r2a
% . In conclusion, this project has advanced the state of the art for in-space additive manufacturing by delivering a validated powder feed system design. The system meets its design objectives and provides a solid foundation for future CSAM development, bringing the vision of reliable, on-demand manufacturing in space one step closer to reality.