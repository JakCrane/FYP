\section{In-Space Manufacturing}
In-space manufacturing (ISM) encompasses techniques for fabricating, assembling and repairing structures and components directly in orbit or on other celestial bodies. Key benefits include dramatically lower launch costs (since only raw feedstock need be lifted), on-demand production of replacement parts, and the ability to build large structures (e.g. habitats or solar arrays) beyond the size limits of launch fairings.

\section{Previous Work}
Research into ISM dates back to the Apollo era. In 1969, the Soyuz 6 mission demonstrated that welding could be performed in vacuum, laying the groundwork for subsequent research and technology demonstrations~\cite{nasa1984welding}. Additive manufacturing (AM) came much later, with the first experiment aboard the ISS occuring in 2014, when a polymer Fused Filament Fabrication (FFF) printer successfully produced samples for stress testing~\cite{Prater2019}. FFF works by extruding thermoplastic wire through a heated nozzle to build parts layer by layer. More recently, in January 2024, ESA deployed a metal 3D printer employing laser-wire direct-energy deposition~\cite{ESA2024Metal3DPrinter}. This involved melting stainless-steel wire onto a print bed to fabricate the metallic components. Wire-based feedstock systems are currently prefered as they simplify material storage and feedrate control in the space environment and much work needs to be done before powder based AM methods can be demonstrated in orbit.

\section{Motivation}
The current line of research exploring Cold Spray Additive Manufacturing (CSAM) in space started with an analysis of different AM methods and how they were influenced by properties of the space environment such as microgravity, thermal constraints and elevated exposure to radiation. Despite the requirement to bring a propellant gas to orbit with the manufacturing facility, the combination of not requiring post processing, flexibility of being able to spray onto unprepared metal surfaces and high deposition rate made CSAM an attractive candidate for further research~\cite{malagowski2019amspace}.

This analysis was later followed up with COSMOS, an Imperial project lead by Dr Ajit Panesar and funded by the UKSA to demonstrate the feasibility CSAM under vacuum conditions. The demonstration was an overall success and samples of the Ti6Al4V deposit were analysed, where it was identified that the porosity of the deposit was higher than expected. This is thought to be because the mass flow rate of the powder was too high, meaning the particles were not accelerated to a high enough velocity and did not deform as expected. Along with the sample porosity, the design of the powder hopper was hypothesised to be unsuitable for a microgravity environment. Therefore, the objectives of this research, outlined below, are underpinned by the need to rectify both the observed high porosity and the hypothesized hopper limitations under microgravity.

\section{Aims and Objectives}
As discussed in \autoref{sec:cold-spray}, the porosity of deposit from CSAM is strongly influenced by the mass flow rate of powder dispensed. This means that any future work aimed at demonstrating the quality of deposit possible from an in-space CSAM system would require a high level of controlability over this parameter. Equally important to the success of ground tests is the authenticity of the conditions the system is emulating. To minimise the uncertainty associated with extrapolating terrrestrial results to operations in orbit, each subsystem must function reliably in the space environment. The main concerns for the feed system are damage when pressurising and depressurising as well as being able to control the powder while under microgravity.

These aims will be achieved through the following objectives:
\begin{itemize}
    \item Investigate the hypothesised issues with the current design.
    \item Design a new architecture to solve the previous issues.
    \item Experimentally investigate it's controllability of mass flow rate in terrestrial conditions.
    \item Investigate piston geometries and their impact on the system dynamics.
    \item Numerically simulate the design in microgravity and terrestrial conditions.
    \item Compare these result to validate the extrapolation into the space environment. 
\end{itemize}


\section{Structure of the Report}
