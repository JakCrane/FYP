\section{In-Space Manufacturing}
In-space manufacturing (ISM) encompasses techniques for fabricating, assembling and repairing structures and components directly in orbit or on other celestial bodies. Key benefits include dramatically lower launch costs (since only raw feedstock needs to be lifted), on-demand production of replacement parts and the ability to build large structures (e.g. habitats or solar arrays) beyond the size limits of launch fairings.

\section{Previous Work}
Research into ISM dates back to the Apollo era. In 1969, the Soyuz 6 mission demonstrated that welding could be performed in vacuum, laying the groundwork for subsequent research and technology demonstrations~\cite{nasa1984welding}. Additive manufacturing (AM) came much later, with the first experiment aboard the ISS occurring in 2014, when a polymer Fused Filament Fabrication (FFF) printer successfully produced samples for stress testing~\cite{Prater2019}. FFF works by extruding thermoplastic wire through a heated nozzle to build parts layer by layer. More recently, in January 2024, ESA deployed a metal 3D printer employing laser-wire direct-energy deposition~\cite{ESA2024Metal3DPrinter}. This involved melting stainless-steel wire onto a print bed to fabricate metallic components. Wire-based feedstock systems are currently preferred, as they simplify material storage and feedrate control in the space environment and much work still needs to be done before powder-based AM methods can be demonstrated in orbit.

\section{Motivation}
The current line of research exploring Cold Spray Additive Manufacturing (CSAM) in space started with an analysis of different AM methods and how they are influenced by properties of the space environment such as microgravity, thermal constraints and elevated exposure to radiation. Despite the requirement to bring a propellant gas to orbit with the manufacturing facility, the combination of not requiring post-processing, the flexibility of being able to spray onto unprepared metal surfaces and a high deposition rate has made CSAM an attractive candidate for further research~\cite{malagowski2019amspace}.

This analysis was later followed up with COSMOS, an Imperial College London project led by Dr Ajit Panesar and funded by the UK Space Agency to demonstrate the feasibility of CSAM under vacuum conditions. The demonstration was an overall success and samples of the Ti6Al4V deposit were analysed. It was identified that the porosity of the deposit was higher than expected. This is thought to be due the mass flow rate of powder being too high, meaning the particles were not accelerated to a sufficient velocity and did not deform as expected. In addition to the sample porosity, the design of the powder hopper was hypothesised to be unsuitable for a microgravity environment. Therefore, the objectives of this research, outlined below, are underpinned by the need to rectify both the observed high porosity and the hypothesised hopper limitations under microgravity.

\section{Aims and Objectives}
As discussed in \autoref{sec:cold-spray}, the porosity of the deposit from CSAM is strongly influenced by the mass flow rate of powder dispensed. This means that any future work aimed at demonstrating the quality of deposit achievable through in-space CSAM would require a system with a high level of controllability over this parameter. Equally important to the success of ground tests is the authenticity of the conditions being emulated. To minimise the uncertainty associated with extrapolating terrestrial results to operations in orbit, each subsystem must function reliably in the space environment. 

These aims will be achieved through the following objectives:
\begin{itemize}
    \item Investigate the hypothesised issues with the current design.
    \item Design a new architecture to solve the previous issues.
    \item Experimentally investigate the controllability of mass flow rate in terrestrial conditions.
    \item Investigate piston geometries and their impact on the system dynamics.
    \item Numerically simulate the design in microgravity and terrestrial conditions.
    \item Compare these result to validate the extrapolation into the space environment. 
\end{itemize}


\section{Structure of the Report}
This report is structured to reflect the logical progression of the research, from contextual grounding through to design, implementation, and evaluation. 

It begins by introducing the broader field of in-space manufacturing and the specific challenges that motivated this project, establishing the aims and objectives that guided the work. To support the technical development, the academic background chapter explores key concepts such as cold spray technology, powder fluidisation behaviour, and relevant approaches to two-phase flow modelling. 

The core of the report focuses on the methodology used to develop and assess the powder feed system—from early design considerations and hardware development to experimental procedures and simulation strategies. This is followed by a discussion of results, which brings together findings from physical testing and numerical modelling to evaluate system performance under both Earth and microgravity conditions. 

Finally, the report concludes by reflecting on how well the project met its objectives and by suggesting avenues for further refinement and investigation. Supporting material, including design drawings, simulation outputs, and analysis code, is provided in the appendices to ensure transparency and reproducibility.